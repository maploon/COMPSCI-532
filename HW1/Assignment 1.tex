\documentclass[10pt,a4paper]{article}

%%%%%%%%% Packages %%%%%%%%
\usepackage{amsmath}
\usepackage{amsthm}
\usepackage{amsfonts}
\usepackage{amssymb}
\usepackage{latexsym}
\usepackage{epsfig}
\usepackage{graphicx}
\usepackage{fancyhdr}
\usepackage{titling}
\usepackage{lipsum}
\usepackage{tikz}
\usepackage{float}
\usepackage[a4paper, total={7in, 9.5in}]{geometry}
\usepackage[linesnumbered,boxruled,commentsnumbered]{algorithm2e}
\usepackage[matrix,tips,graph,curve]{xy}

%%%%%%%%%%%%%%%%%%%%%%%%%%%%%%%%%%
\makeatletter
\@addtoreset{equation}{section}
\makeatother

%%%%%%%%%%%%%%%%%%%%%%%%%%%%%%%%%%

\renewcommand{\theequation}{\thesection.\arabic{equation}}

\theoremstyle{plain}
\newtheorem{theorem}[equation]{Theorem}
\newtheorem{corollary}[equation]{Corollary}
\newtheorem{lemma}[equation]{Lemma}
\newtheorem{proposition}[equation]{Proposition}
\newtheorem{conjecture}[equation]{Conjecture}
\newtheorem{fact}[equation]{Fact}
\newtheorem{facts}[equation]{Facts}
\newtheorem*{theoremA}{Theorem A}
\newtheorem*{theoremB}{Theorem B}
\newtheorem*{theoremC}{Theorem C}
\newtheorem*{theoremD}{Theorem D}
\newtheorem*{theoremE}{Theorem E}
\newtheorem*{theoremF}{Theorem F}
\newtheorem*{theoremG}{Theorem G}
\newtheorem*{theoremH}{Theorem H}

\theoremstyle{definition}
\newtheorem{definition}[equation]{Definition}
\newtheorem{definitions}[equation]{Definitions}

\newtheorem{remark}[equation]{Remark}
\newtheorem{remarks}[equation]{Remarks}
\newtheorem{exercise}[equation]{Exercise}
\newtheorem{example}[equation]{Example}
\newtheorem{examples}[equation]{Examples}
\newtheorem{notation}[equation]{Notation}
\newtheorem{question}[equation]{Question}
\newtheorem{assumption}[equation]{Assumption}
\newtheorem*{claim}{Claim}
\newtheorem{problem}{Problem}
\newtheorem*{problem*}{Problem}

\theoremstyle{remark}
\newtheorem{answer}{Answer}
\newtheorem*{answer*}{Answer}

%%%%%% letters %%%%
\newcommand{\fA}{\mathfrak{A}}
\newcommand{\fB}{\mathfrak{B}}
\newcommand{\fC}{\mathfrak{C}}
\newcommand{\fD}{\mathfrak{D}}
\newcommand{\fE}{\mathfrak{E}}
\newcommand{\fF}{\mathfrak{F}}
\newcommand{\fG}{\mathfrak{G}}
\newcommand{\fH}{\mathfrak{H}}
\newcommand{\fI}{\mathfrak{I}}
\newcommand{\fJ}{\mathfrak{J}}
\newcommand{\fK}{\mathfrak{K}}
\newcommand{\fL}{\mathfrak{L}}
\newcommand{\fM}{\mathfrak{M}}
\newcommand{\fN}{\mathfrak{N}}
\newcommand{\fO}{\mathfrak{O}}
\newcommand{\fP}{\mathfrak{P}}
\newcommand{\fQ}{\mathfrak{Q}}
\newcommand{\fR}{\mathfrak{R}}
\newcommand{\fS}{\mathfrak{S}}
\newcommand{\fT}{\mathfrak{T}}
\newcommand{\fU}{\mathfrak{U}}
\newcommand{\fV}{\mathfrak{V}}
\newcommand{\fW}{\mathfrak{W}}
\newcommand{\fX}{\mathfrak{X}}
\newcommand{\fY}{\mathfrak{Y}}
\newcommand{\fZ}{\mathfrak{Z}}
%%%%%%%%%%%%%%%%%%%%%%%%%%%%%%
\newcommand{\fa}{\mathfrak{a}}
\newcommand{\fb}{\mathfrak{b}}
\newcommand{\fc}{\mathfrak{c}}
\newcommand{\fd}{\mathfrak{d}}
\newcommand{\fe}{\mathfrak{e}}
\newcommand{\ff}{\mathfrak{f}}
\newcommand{\fg}{\mathfrak{g}}
\newcommand{\fh}{\mathfrak{h}}
\newcommand{\ffi}{\mathfrak{i}} %%\fi is defined
\newcommand{\fj}{\mathfrak{j}}
\newcommand{\fk}{\mathfrak{k}}
\newcommand{\fl}{\mathfrak{l}}
\newcommand{\fm}{\mathfrak{m}}
\newcommand{\fn}{\mathfrak{n}}
\newcommand{\fo}{\mathfrak{o}}
\newcommand{\fp}{\mathfrak{p}}
\newcommand{\fq}{\mathfrak{q}}
\newcommand{\fr}{\mathfrak{r}}
\newcommand{\fs}{\mathfrak{s}}
\newcommand{\ft}{\mathfrak{t}}
\newcommand{\fu}{\mathfrak{u}}
\newcommand{\fv}{\mathfrak{v}}
\newcommand{\fw}{\mathfrak{w}}
\newcommand{\fx}{\mathfrak{x}}
\newcommand{\fy}{\mathfrak{y}}
\newcommand{\fz}{\mathfrak{z}}
%%%%%%%%%%%%%%%%%%%%%%%%%%%%%%%
\newcommand{\sA}{\mathcal{A}\,}
\newcommand{\sB}{\mathcal{B}\,}
\newcommand{\sC}{\mathcal{C}}
\newcommand{\sD}{\mathcal{D}\,}
\newcommand{\sE}{\mathcal{E}\,}
\newcommand{\sF}{\mathcal{F}\,}
\newcommand{\sG}{\mathcal{G}\,}
\newcommand{\sH}{\mathcal{H}}
\newcommand{\sI}{\mathcal{I}\,}
\newcommand{\sJ}{\mathcal{J}\,}
\newcommand{\sK}{\mathcal{K}\,}
\newcommand{\sL}{\mathcal{L}\,}
\newcommand{\sM}{\mathcal{M}\,}
\newcommand{\sN}{\mathcal{N}}
\newcommand{\sO}{\mathcal{O}}
\newcommand{\sP}{\mathcal{P}\,}
\newcommand{\sQ}{\mathcal{Q}\,}
\newcommand{\sR}{\mathcal{R}}
\newcommand{\sS}{\mathcal{S}}
\newcommand{\sT}{\mathcal{T}\,}
\newcommand{\sU}{\mathcal{U}\,}
\newcommand{\sV}{\mathcal{V}\,}
\newcommand{\sW}{\mathcal{W}\,}
\newcommand{\sX}{\mathcal{X}\,}
\newcommand{\sY}{\mathcal{Y}\,}
\newcommand{\sZ}{\mathcal{Z}\,}
%%%%%%%%%%%%%%%%%%%%%%%%%%%%%%%
\newcommand{\IA}{\mathbb{A}}
\newcommand{\IB}{\mathbb{B}}
\newcommand{\IC}{\mathbb{C}}
\newcommand{\ID}{\mathbb{D}}
\newcommand{\IE}{\mathbb{E}}
\newcommand{\IF}{\mathbb{F}}
\newcommand{\IG}{\mathbb{G}}
\newcommand{\IH}{\mathbb{H}}
\newcommand{\II}{\mathbb{I}}
\newcommand{\IK}{\mathbb{K}}
\newcommand{\IL}{\mathbb{L}}
\newcommand{\IM}{\mathbb{M}}
\newcommand{\IN}{\mathbb{N}}
\newcommand{\IO}{\mathbb{O}}
\newcommand{\IP}{\mathbb{P}}
\newcommand{\IQ}{\mathbb{Q}}
\newcommand{\IR}{\mathbb{R}}
\newcommand{\IS}{\mathbb{S}}
\newcommand{\IT}{\mathbb{T}}
\newcommand{\IU}{\mathbb{U}}
\newcommand{\IV}{\mathbb{V}}
\newcommand{\IW}{\mathbb{W}}
\newcommand{\IX}{\mathbb{X}}
\newcommand{\IY}{\mathbb{Y}}
\newcommand{\IZ}{\mathbb{Z}}
%%%%%%%%%%%%%%%%%%%%%%%%%%%%%%
\newcommand{\tA}{\mathrm {A}}
\newcommand{\tB}{\mathrm {B}}
\newcommand{\tC}{\mathrm {C}}
\newcommand{\tD}{\mathrm {D}}
\newcommand{\tE}{\mathrm {E}}
\newcommand{\tF}{\mathrm {F}}
\newcommand{\tG}{\mathrm {G}}
\newcommand{\tH}{\mathrm {H}}
\newcommand{\tI}{\mathrm {I}}
\newcommand{\tJ}{\mathrm {J}}
\newcommand{\tK}{\mathrm {K}}
\newcommand{\tL}{\mathrm {L}}
\newcommand{\tM}{\mathrm {M}}
\newcommand{\tN}{\mathrm {N}}
\newcommand{\tO}{\mathrm {O}}
\newcommand{\tP}{\mathrm {P}}
\newcommand{\tQ}{\mathrm {Q}}
\newcommand{\tR}{\mathrm {R}}
\newcommand{\tS}{\mathrm {S}}
\newcommand{\tT}{\mathrm {T}}
\newcommand{\tU}{\mathrm {U}}
\newcommand{\tV}{\mathrm {V}}
\newcommand{\tW}{\mathrm {W}}
\newcommand{\tX}{\mathrm {X}}
\newcommand{\tY}{\mathrm {Y}}
\newcommand{\tZ}{\mathrm {Z}}
%%%%%%% macros %%%%%

%% my definitions %%%

\newcommand{\End}{\mathrm{End}}
\newcommand{\tr}{\mathrm{tr}}
%%\newcommand{\ind}{\mathrm{ind}}

\renewcommand{\index}{\mathrm{index \,}}
\newcommand{\Hom}{\mathrm{Hom}}
\newcommand{\Aut}{\mathrm{Aut}}
\newcommand{\Trace}{\mathrm{Trace}\,}
\newcommand{\Res}{\mathrm{Res}\,}
\newcommand{\rank}{\mathrm{rank}}
%%\renewcommand{\dim}{\mathrm{dim}}

\renewcommand{\deg}{\mathrm{deg}}
\newcommand{\spin}{\rm Spin}
\newcommand{\Spin}{\rm Spin}
\newcommand{\erfc}{\rm erfc\,}
\newcommand{\sgn}{\rm sgn\,}
\newcommand{\Spec}{\rm Spec\,}
\newcommand{\diag}{\rm diag\,}
\newcommand{\Fix}{\mathrm{Fix}}
\newcommand{\Ker}{\mathrm{Ker \,}}
\newcommand{\Coker}{\mathrm{Coker \,}}
\newcommand{\Sym}{\mathrm{Sym \,}}
\newcommand{\Hess}{\mathrm{Hess \,}}
\newcommand{\grad}{\mathrm{grad \,}}
\newcommand{\Center}{\mathrm{Center}}
\newcommand{\Lie}{\mathrm{Lie}}
\newcommand{\ch}{\rm ch} %% Chern Character
\newcommand{\rk}{\rm rk}
\newcommand{\sign}{\rm sign}
\renewcommand\dim{{\rm dim\,}}
\renewcommand\det{{\rm det\,}}
\newcommand{\dimKrull}{{\rm Krulldim\,}}
\newcommand\Rep{\mathrm{Rep}}
\newcommand\Hilb{\mathrm{Hilb}}
\newcommand\vol{\mathrm{vol}}
\newcommand\QED{\hfill $\Box$} %%{\bf QED}}
\newcommand\Pf{\nonintend{\em Proof. }}
\newcommand\reals{{\mathbb R}}
\newcommand\complexes{{\mathbb C}}
\renewcommand\i{\sqrt{-1}}
\renewcommand\Re{\mathrm Re}
\renewcommand\Im{\mathrm Im}
\newcommand\integers{{\mathbb Z}}
\newcommand\quaternions{{\mathbb H}}
\newcommand\iso{{\cong}}
\newcommand\tensor{{\otimes}}
\newcommand\Tensor{{\bigotimes}}
\newcommand\union{\bigcup}
\newcommand\onehalf{\frac{1}{2}}
%%\newcommand\Sym[1]{{Sym^{#1}(\complexes^2)}}
\newcommand\lie[1]{{\mathfrak #1}}
\renewcommand\fk{\mathfrak{K}}
\newcommand\smooth{\mathcal{C}^{\infty}}
\newcommand\trivial{{\mathbb I}}
\newcommand\widebar{\overline}

%%%%%Delimiters%%%%

\newcommand{\<}{\langle}
\renewcommand{\>}{\rangle}

%\renewcommand{\(}{\left(}
%\renewcommand{\)}{\right)}


%%%% Different kind of derivatives %%%%%
\newcommand{\delbar}{\bar{\partial}}
\newcommand{\pdu}{\frac{\partial}{\partial u}}
%\newcommand{\pd}[1][2]{\frac{\partial #1}{\partial #2}}

%%%%% Arrows %%%%%
\newcommand{\ra}{\rightarrow}                   % right arrow
%\newcommand{\lra}{\longrightarrow}              % long right arrow
%\renewcommand{\la}{\leftarrow}                    % left arrow
%\newcommand{\lla}{\longleftarrow}               % long left arrow
%\newcommand{\ua}{\uparrow}                     % long up arrow
%\newcommand{\na}{\nearrow}                      %  NE arrow
%\newcommand{\llra}[1]{\stackrel{#1}{\lra}}      % labeled long right arrow
%\newcommand{\llla}[1]{\stackrel{#1}{\lla}}      % labeled long left arrow
%\newcommand{\lua}[1]{\stackrel{#1}{\ua}}      % labeled  up arrow
%\newcommand{\lna}[1]{\stackrel{#1}{\na}}      % labeled long NE arrow

\newcommand{\into}{\hookrightarrow}
\newcommand{\tto}{\longmapsto}
\def\llra{\longleftrightarrow}

\def\d/{/\mspace{-6.0mu}/}
\newcommand{\git}[3]{#1\d/_{\mspace{-4.0mu}#2}#3}
\newcommand\zetahilb{\zeta_{{\text{Hilb}}}}
\def\Fy{\sF \mspace{-3.0mu} \cdot \mspace{-3.0mu} y}
\def\tv{\tilde{v}}
\def\tw{\tilde{w}}
\def\wt{\widetilde}
\def\wtilde{\widetilde}
\def\what{\widehat}

%%%%%%%%%%%%%%%%%%% Mark's definitions %%%%%%%%%%%%%%%%%%%%

\newcommand{\frakg}{\mbox{\frakturfont g}}
\newcommand{\frakk}{\mbox{\frakturfont k}}
\newcommand{\frakp}{\mbox{\frakturfont p}}
\newcommand{\q}{\mbox{\frakturfont q}}
\newcommand{\frakn}{\mbox{\frakturfont n}}
\newcommand{\frakv}{\mbox{\frakturfont v}}
\newcommand{\fraku}{\mbox{\frakturfont u}}
\newcommand{\frakh}{\mbox{\frakturfont h}}
\newcommand{\frakm}{\mbox{\frakturfont m}}
\newcommand{\frakt}{\mbox{\frakturfont t}}
\newcommand{\G}{\Gamma}
\newcommand{\g}{\gamma}
\newcommand{\fraka}{\mbox{\frakturfont a}}
\newcommand{\db}{\bar{\partial}}
\newcommand{\dbs}{\bar{\partial}^*}
\newcommand{\p}{\partial}

%%%%%%%%%%%%% new definitions for the positive mass paper %%%%%%%%%

\newcommand{\sperp}{{\scriptscriptstyle \perp}}

%%%%%%%% My definitions %%%%%%%%

\newcommand{\norm}[1]{\left\lVert#1\right\rVert}

%%%%%%%% Page Layout %%%%%%%%

\linespread{1.065}

\newcommand{\subtitle}[1]{%
  \posttitle{%
    \par\end{center}
    \begin{center}\large#1\end{center}
    \vskip0.5em}%
}


%%%%%%%%%%%%%%%%%%%%%%%%%%%%%%%%%%%%%%%%%%%%%%%%%%%%%%%%%%%%%%%

%
\begin{document}
%

%%%%%%%% Title %%%%%%%
\title{COMPSCI 532}
\subtitle{Assignment 1}
\author{Feng Gui}
\date{\today}

%%%%%%%% Headers and Footers %%%%%%%%

\fancypagestyle{plain}{%
  \renewcommand{\headrulewidth}{0pt}
  \fancyhf{}%
  \rfoot{PAGE \thepage}
}

\pagestyle{plain}

\makeatletter
\let\runlhead\@author
\let\runrhead\@title
\makeatother

\renewcommand{\headrulewidth}{1.5pt}
\lhead{Assignment 1} %%%% Use \runlhead to put author on left header.
\chead{}
\rhead{\runrhead}

\lfoot{}
\cfoot{}

%%%%%%%% Make Title %%%%%%%%

\maketitle

%%%%%%%% Body %%%%%%%%


%% problem 1
\begin{problem}
\end{problem}

\begin{answer*} \hfill

Suppose $y\in \IR^n$ is the center of a ball that lies indide $P$. Then we know the distance from $y$ to hyperplane $a_i^T x = b_i$ is
\[r_i = \frac{|a_i^T(y-x)|}{\norm{a_i}} = \frac{|a_i^T y-b_i|}{\norm{a_i}}\]

Since $y$ lies inside the polygon $P$, $a_i^Ty \leq b_i$.  Hence $r_i = \frac{b_i-a_i^T y}{\norm{a_i}}$. Rewrite the equation we then get
\[a_i^Ty+\norm{a_i}r_i = b_i\]

For a ball centered at $y$ inside $P$, the largest radius is $r = \min\{r_1,r_2,...,r_m\}$.

Therefore, for a particular $y\in P$, we need to find the maximum of $r$ restrict by $a_i^Ty+\norm{a_i}r \leq b_i$

If we want to find the largest ball inscribed in $P$, we just need to vary $y\in P$. So the largest ball is the optimal solution to linear system
\[\max\, r \quad s.t. \quad a_i^Ty+\norm{a_i}r \leq b_i, \quad y,r \geq 0\]

\end{answer*}


%% problem 2
\begin{problem}
\end{problem}

\begin{answer*} \hfill

First we show that $U = \{x\in \IR^n\vert \norm{x}_\infty \leq \rho\}$ is just a solid "cube" in $\IR^n$. Let $x = (x_1,x_2,...,x_n)$. Without the loss of generality, assume $|x_1| = \max\{|x_1|,|x_2|,...,|x_n|\}$. Now notice that,
\begin{align*}
\norm{x}_\infty &= \lim_{k\ra \infty} \left({x_1}^k+\cdots+{x_n}^k\right)^{\frac{1}{k}} \\
&= |x_1| \lim_{k\ra\infty} \left(1+{\alpha_2}^k+\cdots+{\alpha_n}^k\right)^\frac{1}{k} \\
&= |x_1| \exp(\lim_{k\ra\infty} \frac{1}{k} \ln \left(1+{\alpha_2}^k+\cdots+{\alpha_n}^k\right)) \\
&= |x_1|
\end{align*}
where $|\alpha_i| = |\frac{x_i}{x_1}| \leq 1$.

Therefore the $\norm{x}_\infty = |x_1| \leq \rho$. Thus we know that $U = [-\rho, \rho]^n$.

Now back to our problem. From the deduction above, we have that $A_i = \Pi_{j = 1}^n [\bar{a}_{ij}-\rho_i,\bar{a}_{ij}+\rho_i]$ where the multiplication is the Cartisian product.

Notice that $a_{i1}x_1 + \cdots + a_{in}x_n \leq (\bar{a}_{i1}+\rho_i)x_1 + \cdots + (\bar{a}_{in}+\rho_i)x_n \leq b_i$ is always true, since $x\geq 0$. Therefore any feasible solution to the linear system
\begin{align*}
\max c^T x \quad s.t. \quad (\bar{a}_{i1}+\rho_i)x_1 + \cdots + (\bar{a}_{in}+\rho_i)x_n \leq b_i \quad \forall i\in\{1,...,m\},\quad x\geq 0
\end{align*}
is also a feasible solution to the original LP and vice versa since the LP above only has contraints in the original LP.

Hence the original LP can be rewrite as an LP with $n$ varibles and $m$ constraints.
\end{answer*}


%% problem 3
\begin{problem}
\end{problem}

\begin{answer*} \hfill

(i) No. Let our LP be
\[\max x_1 + x_2\quad s.t. \quad x_1+x_2 \leq 1, \quad x_1,x_2 \geq 0\]

Then the standard form is
\[\max x_1 + x_2\quad s.t. \quad x_1+x_2+x_3 = 1, \quad x_1,x_2,x_3 \geq 0\]

Notice that $n-m = 2$. Therefore there's no vertices in the feasible region that has more than two variables being $0$. Hence there's no degenerate vertices in $F$. However the optimal solution is obviously not unique. Both $x_1=1,x_2=0$ and $x_1 = 0,x_2=1$ are optimal solutions.

(ii) Yes. It is possible.

Suppose we have LP tablau
$$
\begin{pmatrix}
1 & -4 & 2 & 0 & 0 & -2 \\
0 & -2 & 4 & 1 & 0 & 4 \\
0 & -2 & 2 & 0 & 1 & 0
\end{pmatrix}
$$

After 1 pivot, it becomes:
$$
\begin{pmatrix}
1 & -2 & 0 & 0 & -1 & -2 \\
0 & -6 & 0 & 1 & -2 & 4 \\
0 & -1 & 1 & 0 & 1/2 & 0 \\
\end{pmatrix}
$$

Notice that the previous one is already the optimal solution. We just couldn't aware of that since the vertex is degenerate. Therefore we did one more pivot operation which end up having same objective value.
\end{answer*}



%% problem 4
\begin{problem}
\end{problem}

\begin{answer*} \hfill

(i) We can view the $st$-flow as the aggregate of several paths $\{p_i\}$, where $p_i = \{p_{i1} = s, p_{i2}, ..., p_{in_i} = t\}$. We can see it this way because of the conservation of flows property on each vertex.  Suppose the flow on path $p_i$ is $x_i$. Then we must have $\sum_i x_i = 1$. One of paths $\{p_i\}$ must have minimal total weight, since there are only finite paths. Without the loss of generality, let it be path $p_1$. Then $x_i$ times the total weight of other paths must greater $x_i$ times the total wieght of $p_1$. Therefore the total weight of the flow $\sum_e w(e)f(e)$ must greater or equal to the total weight of $p_1$ as $\sum_i x_i = 1$. Hence minimizing $\sum_e w(e)f(e)$ is equivalent to finding the shortest path from $s$ to $t$. \QED

(ii) Define $O(v) = \{(v,u)\in E\vert u\in V\}$ and $I(v) = \{(w,v)\in E \vert w\in V\}$. Let $a_e = w(e)$ and $f_e$. Then the linear program is:
\begin{align*}
\min \quad &\sum_{e\in E}a_ef_e \\
s.t. \quad &\sum_{e\in O(s)}f_e-\sum_{e\in I(s)}f_e = 1 \\
& \sum_{e\in O(v)}f_e-\sum_{e\in I(v)}f_e = 0,\quad \forall v\in V\setminus \{s,t\} \\
&\sum_{e\in O(t)}f_e-\sum_{e\in I(t)}f_e = -1 \\
& f_e\geq 0
\end{align*}


(iii) Dual linear program is:
\begin{align*}
\max \quad & y_s-y_t \\
s.t. \quad & y_u-y_v \leq a_{(u,v)} \quad \forall (u,v)\in E
\end{align*}
\end{answer*}


\newpage
%% Problem 5
\begin{problem}
\end{problem}

\begin{answer*} \hfill

Let $x_{uv} = x_{(u,v)} = 1$ if $(u,v)\in E$ is in the perfect match, and let it be $0$ otherwise. Let $c_uv = c_{(u,v)}$ be the weight for edge $(u,v)$ and $N(v)$ be the set of neighbors of $v$. Then we can formulate perfect matching as a linear program:
\begin{align*}
\min \quad & \sum_{(u,v)\in E} c_{uv}x_{uv} \\
s.t. \quad & \sum_{v\in N(u)} x_{uv} = 1, \quad \forall u\in V=A\cup B \\
& x_{uv} \geq 0 \\
& x_{uv} \in \{0,1\} \\
\end{align*}

We will show that if there exists a perfect matching, the last constraint can be ignored. In otherword, the optimal solution for this LP is always integral if a perfect matching exist. If we ignore the integer constraint, we can formulate its dual LP:
\begin{align*}
\min \quad & \sum_{v\in V=A\cup B} y_v \\
s.t. \quad & y_u+y_v \leq c_{uv},\quad\forall (u,v)\in E
\end{align*}

Now suppose $G$ has a perfect matching. There must be a minimal weight perfect matching since there only finite number of edges. Notice that if there's a fractional edge in the optimal solution, we can trace a cycle along this edge. Suppose $x_{uv}$ is fractional, then there exists $w\in N(v)$ such that $0 < x_{vw} < 1$. In this way, we can trace the path $p = \{u,v,w,...\}$. Since there are finite vertices in the graph, there must be a cycle in the graph (the cycle we find this way may not end at $u$).

Notice that a cycle in a bipartite graph $G = (A\cup B, E)$ must have $2k$ edges and $2k$ vertices evenly divided into two parts. So if we rename $p = \{p_1,p_2,...,p_{2k}\}$, we will have two different perfect matching of this subgraph. One is $M_1 = \{(p_1,p_2), (p_3,p_4),...,(p_{2k-1},p_{2k}\}$ and the other is $M_2 = \{(p_2,p_3),...,(p_{2k},p_1)\}$. One of these two must have smaller total weight. Without the loss of generality, assume $M_1$ has smaller total weight. Then we let
\begin{align*}
x_{(p_{2i-1},p_{2i})}' = x_{(p_1,p_2)}+\varepsilon \\
x_{(p_{2i},p_{2i+1})}' = x_{(p_1,p_2)}-\varepsilon
\end{align*}
for $i = 1,2,...,k$ with $p_{2k+1} = p_1$ and any $\varepsilon>0$. For the other edges we keep $x_e' = x_e$.

We can find a $\varepsilon$ small enough so that the new solution is feasible. Since it changes $\varepsilon\cdot c(M_2)$ to $\varepsilon\cdot c(M_1)$ where $c(M)$ is the total weight of the matching, it decrease the objective function. But the original solution we chose is optimal. By contradiction, the optimal solution must be integral.
\end{answer*}

\end{document}
